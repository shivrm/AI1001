\documentclass{article}
\usepackage[preprint]{neurips_2024}

\author{Shivram S \\ \texttt{ai24btech11031@iith.ac.in}}
\title{AI1001 - Assignment 5}

\begin{document}
\maketitle

The sharp rise in computing power over the years has given
us several new tools for scientific computation. Systems such
as Magma and Mathematica are used for symbolic computation.
SAT (satisfiability) and SMT (satisfiability modulo theory)
solvers are used for making logical deductions from hypotheses.

Many promising uses of computers are also emerging, such as
machine learning, proof assistants, and large language models (LLMs).

\section{Proof Assistants in Mathematics}

The first machine-generated proof was a proof of the
\textbf{four-color theorem}. This proof involved reducing
a graph to one of a few thousand cases, and showing that
each of these cases was four-colorable. Showing the
colourability of each of these cases was tedious, and
a computer greatly simplified the process.

Proof assistants are programs that verify the correctness
of a proof using logical deduction based on predefined rules.
\textbf{Lean} and \textbf{Coq} are two widely used proof assistants.
They include a library of pre-proved lemmas which simplify
the creation of new proofs

Proof assistants can be used to explore proofs that are beyond
human intuition, existing literature, and connections to
other ways of thinking. They can be used in tandem with AI to 
find new proof strategies and to automatically generate simple proofs.


\section{Machine Learning in Mathematics}

Machine learning has been used to discover unintuitive mathematical
relations. For example, it was used in knot theory to discover the
relation between the hyperbolic invariants and the signature of a knot.

The authors trained a neural network on a database of two million knots
to predict the signature of a knot based on its hyperbolic invariants.
Then, based on a \textbf{saliency analysis}
(checking the sensitivity of the network to perturbations of a parameter),
they determined that the signature depends mostly on three out of the
three dozen parameters. This allowed them to propose a conjecture that
explained the relation.

\section{LLMs in Mathematics}

LLMs show promise in suggesting proof strategies and generating simple
proofs. An LLM can be connected to traditional tools like WolframAlpha to boost
its power to do mathematics. By making an LLM produce output in a proof
assistant language, we can ensure some level of correctness. Conversing
with the model can also stimulate mathematical thought indirectly
by suggesting proof approaches.

Tools such as GitHub Copilot have shown moderate productivity increase
when writing code. More advanced tools such as Lean Copilot are in development
to suggest proofs for short mathematical statements, using proof assistants
to filter out hallucinations and provide feedback to the AI.

\end{document}
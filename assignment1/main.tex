\documentclass{article}
\usepackage[preprint]{neurips_2024}

\author{Shivram S \\ \texttt{ai24btech11031@iith.ac.in}}
\title{AI1001 - Assignment 1}

\begin{document}
\maketitle

\section{AI boom}

The AI boom is an ongoing period of rapid progress in AI. It started in the late 2010s
and was catalyzed by the usage of artificial neural networks and deep learning. The AI
boom has had cultural, economic and social impact and has led to artificial intelligence
becoming a widely prominent topic of popular discussion. Concerns such as impersonation of
humans, intellectual property rights, energy usage, threats to job security, biases in
training data, and risk of human extinction have been raised in response to it.

The AI boom has led to the development of programs such as AlexNet (image recognition),
AlphaZero (playing board games) and AlphaFold (protein folding), as well as generative
models for text, image and music generation. These programs involve techniques such as
generative adversarial networks (GANs), diffusion models and transformer architectures.
Established companies such as Apple, Amazon, Google and Microsoft have dominated the
market using their vast resources of cloud infrastructure, AI chips and computing power.
Companies such as Nvidia who manufacture GPUs for AI training and usage have seen a huge
growth in market cap.

\section{AI effect}

The AI effect a phenomenon where the tasks that an AI program can perform successfully 
re no longer considered as AI, but simply as automations. The notion of ``intelligence''
is redefined to exclude whatever has been performed successfully. The term was coined by
John McCarthy, who put it simply as "As soon as it works, no one calls it AI anymore."
Michael Kearns suggests that people subconsciously feel the need to be special and by
discounting artificial intelligence, they can continue to feel unique and special.

The AI effect happens in diverse fields such as agriculture, hospitality, and marketing,
and computer chess and leads to AI and AI researchers being underappreciated. This led
to AI researches in the 1990s avoiding the term ``artificial intelligence.'' The effect
can be mitigated by underselling AI and making it less conspicuous. According to the
Bulletin of Atomic Scientists, the AI effect is a worldwide strategic threat, and the
failure to recognise AI can undermine the recognition of security threats.

\end{document}
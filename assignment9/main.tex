\documentclass{article}
\usepackage[preprint]{neurips_2024}
\usepackage{amsmath}

\DeclareMathOperator*{\argmin}{argmin}

\author{Shivram S \\ \texttt{ai24btech11031@iith.ac.in}}
\title{AI1001 - Assignment 9}

\begin{document}
\maketitle

Our goal in machine learning is to select a hypothesis that will
optimally fit future examples. We make the assumption that our 
future examples will be like the past, and follow the same probability
distribution (\textbf{stationarity} assumption), and that every example
is independent of previous examples. Examples that satisfy these
equations are independent and identically distributed (i.i.d.).

\begin{eqnarray*}    
P(E_j) = E(E_{j+1}) = P(E_{j+2}) = \dots \\
P(E_j) = P(E_j \mid E_{j-1}, E_{j-2}, \dots)
\end{eqnarray*}

We can say that the optimal hypothesis is the one that minimizes the
\textbf{error rate}: the proportion of times that $h(x) \ne y$ for an
$(x,y)$. This is done by measuring the model's performance on a
\textbf{test set} of examples. To prevent the model from peeking at the
test answers, we split the examples we have into a \textbf{training set}
and a \textbf{test set}.

Our model class might have some parameters (called \textbf{hyperparameters})
and we might want to find the optimal values for them. Tuning the 
hyperparameters by measuring error rates on the test set is also considered as peeking.
So, we can divide our data data into a \textbf{training set} to train the
model, a \textbf{validation set} to tune the hyperparameters, and a
\textbf{test set} to do an unbiased evaluation.

If we don't have enough data, we can use \textbf{k-fold cross validation}.
We split the data into $k$ equal subsets. We perform $k$ rounds of
learning. On each round, we use $1/k$ of the data as a validation set
and the rest as the training set. The average test score of the
$k$ rounds is used. Popular values of $k$ are 5 and 10, enough to
give a better estimate at a cost of 5 to 10 times longer computation
time. When $k=n$ we call it \textbf{leave-one-out cross validation} (LOOCV).

\section{Model Selection}

Part of model selection is qualitative and subjective. Based on what
we know about the problem, we might prefer some model classes over others.
Then we can qualitatively select the best model class based on
performance on the validation data set.

We can measure the complexity of a model based on attributes such as 
the number of nodes in a decision tree or the number of neural network
parameters. The training set error approaches zero as the complexity
increases, but the validation error starts to increase after some point due to overfitting.
Some model classes, such as decision trees, never recover from overfitting.
Others classes such as deep neural networks, kernel machines, etc.
can use the larger capacity fit a larger number of suitable
representations, hence the validation error tends to decrease as the 
capacity increases.

Model classes start to overfit as the capacity approaches the point of
interpolation, which is when the model exactly fits all the training data.
This is because the model's capacity is concentrated on the training
examples and the remaining capacity is allocated in a way that is not 
representative of the training data.

\section{Loss Functions}

In machine learning, it is traditional to express the model's
performance in terms of a \textbf{loss function} that needs to be
minimized. The loss function $L(x, y, \hat y)$ is the amount of
utility lost by predicting $h(x) = \hat y$ when the correct answer is
$f(x) = y$. We can often use a simplified version $L(y, \hat y)$ that is
independent of $x$.

\[
L(x, y, \hat y) = Utility(\text{result of using $y$ given $x$})
- Utility(\text{result of using $\hat y$ given $x$})
\]

One misclassification might be worse than another.
For example, if a spam-detection algorithm classifies a spam email
as non-spam, then it's just a minor annoyance, but if a non-spam
email is classified as spam, then the user might miss an important
message. Hence, we might want to give a larger value to $L(nospam,spam)$
than to $L(spam, nospam)$.

We consider smaller errors to be better than larger ones. We can
quantititatively implement this using loss functions such as $L_1$ loss
and $L_2$ loss. For discrete-valued inputs, we can use the
$L_{0/1}$ loss function.

\begin{align*}
    L_1(y, \hat y) &= \lvert y - \hat y \rvert \\
    L_2(y, \hat y) &= (y - \hat y)^2 \\
    L_{0/1}(y, \hat y) &= \text{0 if $y = \hat y$ else 1}
\end{align*}

To compute the expected loss over all input-output pairs, we define
a probability distribution $P(X, Y)$ over the examples. Then we can define
the expected \textbf{generalization loss} over the set of examples
$\mathcal E$ for a hypothesis $h$ as:

\[
GenLoss_L(h) = \sum_{(x, y) \in \mathcal E} L(y, h(x)) \cdot P(x, y)
\]

The best hypothesis $h^*$ is the one that minimizes expected
generalization loss:
\[
h^* = \argmin_{h \in \mathcal H} GenLoss_L(h)
\]

Since $P(x, y)$ is not known in most cases, the learning agent can
only estimate generalization loss with \textbf{empirical loss} on a set
of examples $E$ of size $N$.

\[
EmpLoss_{L,E} (h) = \sum_{(x, y) \in E} L(y, h(x)) \frac{1}{N}
\]

The estimated best hypothesis $\hat h^*$ is the one with the minimum
empirical loss:
\[ 
\hat h^* = \argmin_{h \in \mathcal H} EmpLoss_{L,E}(h).
\]

$\hat h^*$ may differ from the true function $f$ due to unrealizability,
variance, noise and computational complexity. A problem is realiizable if the
hypothesis space $\mathcal H$ actually contains the true function $f$. $f$ may be
non-deterministic or noisy - it may return different values of $f(x)$ for the same
value of $x$.

The early years of machine learning concentrated on \textbf{small-scale learning} where
the number of training example ranges from dozens to the low thousands, and
generalization loss usually came from approximation error of not having $f$ in the 
hypothesis space. Recently there has been a shift towards \textbf{large-scale learning}
with millions of examples, where the generalization loss is dominated by limits of
computation.

\section{Regularization}

Complicated hypotheses have a tendency to overfit. \textbf{Regularization} is the
practice of penalizing complex hypotheses. This is done using a
\textbf{regularization function} which depends on the hypothesis space. For example,
for polynomials, a choice of regularization function may be the sum of squares of
coefficients. Taking regularization into account, we can define the total cost as:

\[
Cost(h) = EmpLoss(h) + \lambda \times Complexity(h)
\]

$\lambda$ is a hyperparameter that serves as a conversion rate between loss and
complexity. However, it is possible to avoid the conversion factor by encoding the
hypothesis as a Turing machine program and counting the number of bits required to
encode the data. The \textbf{minimum description length} hypothesis minimizes the total
number of bits required.

Models can also be simplified by reducing the dimensions that they work
with. \textbf{Feature selection} can be performed to discard irrelevant attributes.

\section{Hyperparameter Tuning}

We also want to select the best values for hyperparameters. The simplest approach is
\textbf{hand-tuning}, where we guess parameter values based on past experience or intuition.
If there are only a small number of possible values, \textbf{grid-search} can be used,
which tries all combinations and sees which performs best on the validation data. If
there are too many combinations, then we might use \textbf{random search} by sampling
some values randomly and repating as long as we are willing to spend the time and
resources.

\textbf{Bayesian Optimization} treats hyperparameter tuning as a machine learning problem
in itself. We think of the vector of hyperparameters, $\textbf{x}$ as an input, and
try to find the function $y = f(\textbf{x})$ which minimizes the loss $y$. Each pair
of $(y, f(\textbf{x}))$ can be used to update our belief about the shape of the
function $f$. We want to trade off exploitation (choosing parameter values near a
previous good result) with exploration (trying novel values).

An alternative to Bayesian optimization is \textbf{population-based-training} (PBT). PBT
first uses random search to train a population of models, each with different
hyperparameter values, then training a second generation whose hyperparameter values
are determined by the best-performing values of the previous generation, plus random
mutation. 




\end{document}
\documentclass{article}
\usepackage[preprint]{neurips_2024}

\title{AI1001 - Assignment 3}
\author{Shivram S \\ \texttt{ai24btech11031@iith.ac.in}}

\begin{document}

\maketitle

The definition of ``intelligence'' varies along two dimensions: human vs. rational and
thought vs. behaviour, each with different methods. The six main
disciplines are natural language processing, knowledge representation, automated reasoning,
machine learning, computer vision, and robotics.

\begin{itemize}
    \item The {\bf Turing Test} (Alan Turing, 1950) was proposed to test the intelligence of a computer.
    A computer passes the test if a human interrogator cannot tell the computer's responses from
    those of a human. 

    \item {\bf Cognitive science} brings together AI and psychology. We can learn about human
    thought through introspection, brain imaging, and psychological experiments. A sufficiently
    precise theory of the mind allows us to express the theory as a computer program.

    \item The {\bf logicist tradition} of artificial intelligence builds on programs that can
    solve any problem written in logical notation (solvers). The uncertainty of the real world
    is filled using \textbf{probability}, allowing rigorous reasoning even with uncertain information.

    \item Computer agents are programs that percieve their environment and operate autonomously
    for extended periods. A {\bf rational agent} acts to achieve the best expected outcome.
\end{itemize}


The rational-agent approach is popular due to its generality and flexibility. The
\textbf{standard model} involves agents that ``do the right thing.''. However, the values
given to the machine may not match our true values (value alignment problem), and
may have negative consequences. For example a chess-playing machine with the sole goal of
winning might try to kill the opponent. We want the agent to be cautious when the objective is
unknown, and ultimately, be \textbf{provably beneficial}. 

The foundations of artificial intelligence are drawn from a large number of disciplines:
\begin{itemize}
    \item {\bf Philosophy}: \textbf{Aristotle's syllogisms} for logical reasoning were an attempt
        to formulate the working of the rational mind. \textbf{Dualism} held that these was a 
        part of the mind exempt from physical laws, and \textbf{materialism} held otherwise.
        The \textbf{principle of induction} is used to reduce problems,
        The actions of an agent are governed either by \textbf{utilitarianism} (best possible outcome)
        or by \textbf{deontological ethics} (agreement with value system).
    
    \item {\bf Mathematics}: \textbf{Probability} generalizes logic to uncertain situations. \textbf{Statistics}
        and \textbf{decision theory} provide a framework for making decisions. \textbf{Game theory} lets us
        account for the actions of other agents. The notion of
        \textbf{satisficing} - making decisions that are ``good enough'', gives a better description
        or actual human behaviour.

    \item {\bf Neuroscience}: The exact working of the human brain is still a great mystery.
        Digital computers operate at a clock rate that is a million times faster than a brain
        but the brain makes up for that with storage and interconnection.
        The development of \textbf{brain-machine interfaces} is promising and may shed light
        on many aspects of neural systems.

    \item {\bf Psychology}: \textbf{Cognitive physchology} views the brain as an information-
        processing device. \textbf{Behaviourism} looks at the response of systems to stimulus.

    \item {\bf Computer Engineering}: Faster computers allow for speedup in AI algorithms.
        \textbf{Quantum computing} might offer a large speedup. Specialized hardware such as
        GPUs, NPUs and wafer scale engine (WSE) have been developed for AI applications.

    \item Artificial intelligence also borrows \textbf{regulatory mechanisms} from {\bf control theory},
        and developments in \textbf{knowledge representation} from {\bf linguistics}.
\end{itemize}

\end{document}